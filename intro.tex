OMICS technologies are becoming increasingly popular for large-scale data
generation in biology and medicine. Recently, there have been major
breakthroughs in the application of workflow systems (e.g.  Galaxy,
gUSE/WS-PGrade) towards automated analysis of biological high-throughput data.
Mass spectrometry instrumentation, the core technology in proteomics, provides
improved sensitivity and throughput for proteomics research. This increasing
speed of data generation and the growing data amount require automated methods
for data processing.

MaxQuant is a widely used tool to analyze proteomics data. In its latest
release it is operated using an interactive graphical user interface (GUI) on
Windows and is not compatible with Linux, prohibiting easy access from the
command line or as a library.

We propose mqrun — a wrapper to integrate MaxQuant workflows into a Linux
environment. While previous work focused on the integration into Windows-based
workflows, most clusters are Linux based.

mqrun is a Python library that controls MaxQuant running on a virtual Windows
server. The configuration for MaxQuant is specified in a json file that aims to
be easy for humans to edit and understand. The supplied json-schema can be used
to generate HTML forms or GUIs to allow for integration into existing workflow
solutions.
