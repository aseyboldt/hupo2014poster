OMICS technologies are becoming increasingly popular for large-scale data
generation in biology and medicine. Recently, there have been major
breakthroughs in the application of workflow systems (e.g.  Galaxy,
gUSE/WS-PGrade) towards automated analysis of biological high-throughput data.
Mass spectrometry instrumentation, the core technology in proteomics, provides
improved sensitivity and throughput for proteomics research. This increasing
speed of data generation and the growing data amount require automated methods
for data processing.

MaxQuant is a widely used tool to analyze proteomics data. In its latest
release it is operated using an interactive graphical user interface (GUI) on
Windows and is not compatible with Linux, prohibiting easy access from the
command line or as a library.

We propose mqrun — a wrapper to integrate MaxQuant workflows into a Linux
environment. While previous work focused on the integration into Windows-based
workflows, most clusters are Linux based. Making MaxQuant available from Linux
broadens its applicability.

mqrun is implemented as a Python library that controls a MaxQuant instance on
a remote Windows server. The configuration for MaxQuant is specified in a json
file that aims to be easy for humans to edit and understand. The supplied
json-schema can be used to generate HTML forms or GUIs to allow for
integration into existing workflow solutions. The Windows machine can be
virtualized on clusters or dedicated servers; only a shared directory between
the Linux process and the Windows machine is needed.
%
%We implemented tests to show mqrun reliably reports failures and crashes to
%the client.
%
%With mqrun MaxQuant can be integrated into automated workflows on a cluster
%which allows for chaining with downstream analysis tools (e.g. R). The textual
%parameter file makes it easier to reproduce results.  By taking advantage of
%cluster resources, many data analyses can be parallelized, increasing
%performance and throughput in proteomics research.
%
